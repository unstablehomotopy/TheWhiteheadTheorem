\documentclass{beamer}
\usetheme{Boadilla}
\setbeamertemplate{footline}{}

\usepackage{ctex} % Chinese characters
\usepackage{fontspec}

\usepackage{hyperref}

\usepackage{mathrsfs}  %\mathscr

%%%%%%%%%%%%%%%%%%%%%%%%%%%%%%% Lean code
% https://lean-lang.org/lean4/doc/syntax_highlight_in_latex.html
\usepackage{fontspec}
% switch to a monospace font supporting more Unicode characters
\setmonofont{JuliaMono}
\usepackage{minted}
% instruct minted to use our local theorem.py
\newmintinline[lean]{lean4.py:Lean4Lexer -x}{bgcolor=white}
\newminted[leancode]{lean4.py:Lean4Lexer -x}{fontsize=\footnotesize}
%\usemintedstyle{tango}  % a nice, colorful theme (problem: red box around some Unicode characters)
\usemintedstyle{xcode}

%%%%%%%%%%%%%%%%%%%%%%%%%%%%%%%


%\usepackage{biblatex}
%\addbibresource{b.bib}

\title{Three Whitehead Theorems and Three Puppe Sequences}
%\subtitle{subtitle}
\author{E.~Dean~Young and Jiazhen~Xia (夏家桢)}
%\institute{Zhejiang University (Department of Computer Science)}
\date{December 15, 2023}


%%%%%%%%%%%%%%%%%%%%%%%%%%%%%%%

\begin{document}


\begin{frame}
\titlepage
\end{frame}

%%%%%%%%%%%%%%%%%%%%%%%%%%%%%%%

\section{The plan}
%\subsection{sub a}

\begin{frame}[fragile]
\frametitle{Two goals}
\begin{block}{The Whitehead Theorem}
\begin{leancode}
          ∀X : D(∞-Grpd₀), ∀Y : D(∞-Grpd₀),
          ∀f : X ⭢ Y, ∀g : X ⭢ Y,
          (∀n : Nat, πₙ f = πₙ g) → f = g
\end{leancode}
\end{block}

\begin{itemize}
	\item $\infty$-$\mathsf{Grpd}_0$ is the category of based connected simplicial sets with the Kan lifting condition.
\end{itemize}

\medskip

\begin{block}{The Puppe Sequence}
\begin{leancode}
          ⬝⬝⬝ ⭢ π₁.obj E₀ ⭢ π₁.obj B₀ ⭢
          π₀.obj ((𝟙 B₀) • ((ω.hom (𝟙 B₀)).hom f)) ⭢
          π₀.obj (E₀) ⭢ π₀.obj(B₀)
\end{leancode}
\end{block}

\begin{itemize}
	\item The homotopy fiber
\end{itemize}

%\medskip
%\begin{itemize}
%	\item 1
%	\item 2
%\end{itemize}
\end{frame}

%\begin{frame}[fragile]
%\frametitle{Puppe Sequence}
%\begin{leancode}
%⬝⬝⬝ ⭢ π₁.obj E₀ ⭢ π₁.obj B₀ ⭢
%π₀.obj ((𝟙 B₀) • ((ω.hom (𝟙 B₀)).hom f)) ⭢
%π₀.obj (E₀) ⭢ π₀.obj(B₀)
%\end{leancode}
%\medskip
%\begin{itemize}
%	\item The homotopy fiber
%\end{itemize}
%\end{frame}

%%%%%%%%%%%%%%%%%%%%%%

\begin{frame}[fragile] % The fragile option is needed when the content contains verbatim.
\frametitle{Definition of $\infty$-category}
  \begin{leancode}
  def horn_filling_condition (X : SSet) (n i : Nat): Prop :=
    ∀ f : Λ[n, i] ⟶ X, ∃ g : Δ[n] ⟶ X,
    f = SSet.hornInclusion n i ≫ g
  
  /-- A simplicial set is called an ∞-category
  if it has the extension property for all inner horn inclusions
  `Λ[n, i] ⟶ Δ[n]`, n ≥ 2, 0 < i < n. -/
  def InfCategory := {X : SSet //
    ∀ (n i : Nat),
    n ≥ 2 ∧ 0 < i ∧ i < n → horn_filling_condition X n i}
    
  #check (inferInstance : Category SSet) -- OK
  #check (inferInstance : Category InfCategory) -- fail
  \end{leancode}
\end{frame}

\begin{frame}[fragile] % The fragile option is needed when the content contains verbatim.
\frametitle{Automatic typeclass inference?}
  \begin{leancode}
  -- instance : Category InfCategory := inferInstance -- ?
  
  -- instance : Category InfCategory := by -- ?
  --   dsimp only [InfCategory]
  --   infer_instance
  
  instance : Category InfCategory where
    Hom X Y := NatTrans X.1 Y.1
    id X := NatTrans.id X.1
    comp α β := NatTrans.vcomp α β

  #check (inferInstance : Category SSet) -- OK
  #check (inferInstance : Category InfCategory) -- OK
  \end{leancode}
\end{frame}

%%%%%%%%%%%%%%%%%%%%%%

\begin{frame}
   \frametitle{Related Work}
   
   \begin{itemize}
   \item Spectral Sequences in Homotopy Type Theory
   	   \begin{itemize}
	   \item Floris van Doorn et al.
	   \item \url{https://github.com/cmu-phil/Spectral}
	   \item Lean 2
	   \end{itemize}
	\item Myers, David Jaz, Hisham Sati, and Urs Schreiber. ``Topological Quantum Gates in Homotopy Type Theory.'' \textit{arXiv preprint arXiv:2303.02382} (2023).
   	   \begin{itemize}
	   \item simulating and verifying topological quantum gates
	   \item Professor Schrieber suggested we formalize the construction of their ``Gauss-Manin connections.''
	   \end{itemize}
   \end{itemize}
\end{frame}

%\begin{frame}[allowframebreaks]
%	\nocite{*}
%       \frametitle{References}
%       %\printbibliography
%\end{frame}


%%%%%%%%%%%%%%%%%%%%%%




\end{document}